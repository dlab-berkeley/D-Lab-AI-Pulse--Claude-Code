\documentclass[aspectratio=169]{beamer}

\usetheme{Madrid}
\usecolortheme{default}

\usepackage{listings}
\usepackage{booktabs}
\usepackage{hyperref}
\usepackage{xcolor}

% Code listing style
\lstset{
    basicstyle=\ttfamily\small,
    breaklines=true,
    backgroundcolor=\color{gray!10},
    keywordstyle=\color{blue},
    commentstyle=\color{green!50!black},
    stringstyle=\color{red!70!black},
    showstringspaces=false
}

\title{Claude Code: AI-Powered CLI for Research}
\subtitle{D-Lab AI Pulse Workshop}
\author{}
\date{}

\begin{document}

\begin{frame}[plain]
    \titlepage
\end{frame}

\begin{frame}{AI in the Browser: The Traditional Approach}
    Most people interact with AI through \textbf{web browsers}:
    \begin{itemize}
        \item ChatGPT, Claude.ai, Gemini, etc.
        \item Convenient for quick questions and conversations
    \end{itemize}

    \vspace{1em}
    \textbf{But for coding and research, browsers have limitations:}
    \begin{itemize}
        \item Must copy/paste code back and forth
        \item AI can't see your actual files or project structure
        \item Memory/Context limits are usually binding
        \item Can only work with one file at a time
        \item You have to manually run every command
    \end{itemize}
\end{frame}

\begin{frame}{A Better Way: Command Line Tools}
    \textbf{CLI (Command Line Interface)} -- A text-based way to interact with your computer

    \vspace{1em}
    CLI tools like \textbf{Claude Code} run directly in your terminal, \textit{inside} your project.

    \vspace{1em}
    \textbf{This changes everything:}
    \begin{itemize}
        \item Direct access to your file system
        \item Can read, create, and modify files
        \item Works within your actual project environment
        \item No copy-pasting code back and forth
        \item Maintains context across your session
    \end{itemize}
\end{frame}

\begin{frame}[fragile]{Using Claude Code}
    \textbf{Let's see it in practice!} 
    
    \vspace{1em}
    
    It might feel awkward at first, since nowadays we don't use terminals often, but it has the same functionalities as the browser.
    
    \vspace{1em}
    Claude Code can then:
    \begin{itemize}
        \item Read your files to understand the context
        \item Write or modify code directly
        \item Run commands and tests
        \item Iterate based on results
    \end{itemize}

    \vspace{1em}
    \textbf{The CLI operates IN your project, not alongside it.}
\end{frame}

\begin{frame}{Browser vs CLI: Key Differences}
    \begin{table}
        \centering
        \begin{tabular}{ll}
            \toprule
            \textbf{Browser Chat} & \textbf{CLI Tool} \\
            \midrule
            Copy/paste code snippets & Directly edits your files \\
            One file at a time & Reads entire codebases \\
            You run the code & Can execute commands for you \\
            Context resets each session & Maintains project memory \\
            General knowledge & Sees YOUR actual code \\
            \bottomrule
        \end{tabular}
    \end{table}

    \vspace{1em}
    \textbf{The CLI operates IN your project, not alongside it.}
\end{frame}

\begin{frame}{When to Use Which?}
    \begin{table}
        \centering
        \begin{tabular}{ll}
            \toprule
            \textbf{Browser Chat} & \textbf{CLI Tools} \\
            \midrule
            Quick conceptual questions & Multi-file projects \\
            Learning new concepts & Iterative development \\
            Simple code snippets & Code that needs testing \\
            Brainstorming ideas & Working with existing codebases \\
            & Repetitive tasks across files \\
            \bottomrule
        \end{tabular}
    \end{table}
\end{frame}

\begin{frame}[fragile]{Installation: Prerequisites}
    \textbf{1. Install Node.js} (required for all platforms):
    \begin{itemize}
        \item Download from \url{https://nodejs.org/} (LTS version recommended)
        \item Or use a package manager: \texttt{brew install node} (macOS)
    \end{itemize}

    \vspace{1em}
    \textbf{2. Windows users: Install WSL} (Windows Subsystem for Linux):
    \begin{lstlisting}
wsl --install
    \end{lstlisting}
    \begin{itemize}
        \item Restart your computer after installation
        \item Run Claude Code from within WSL, not PowerShell/CMD
    \end{itemize}
	\vspace{1em}
    \textbf{Use Claude to Debug/Ask for Help!}
\end{frame}

\begin{frame}[fragile]{Installation: Claude Code}
    \textbf{Install Claude Code} (macOS/Linux/WSL):
    \begin{lstlisting}
npm install -g @anthropic-ai/claude-code
    \end{lstlisting}

    \textbf{First run:}
    \begin{lstlisting}
claude
    \end{lstlisting}

    \textbf{Authentication options:}
    \begin{itemize}
        \item \textbf{Claude Pro/Team subscription} -- Login with your browser (recommended)
        \item Anthropic API key -- For programmatic access
    \end{itemize}

    \vspace{0.5em}
    \textbf{Full docs:} \url{https://docs.anthropic.com/en/docs/claude-code}
\end{frame}

\begin{frame}[fragile]{Permissions and Security}
    \textbf{By default, Claude Code asks permission} before:
    \begin{itemize}
        \item Editing or creating files
        \item Running shell commands
        \item Accessing external resources
    \end{itemize}

    \vspace{0.5em}
    \textbf{This keeps you in control} -- you can review each action before it happens.

    \vspace{0.5em}
    \textbf{Customize allowed actions} in \texttt{.claude/settings.json}:
    \begin{lstlisting}[basicstyle=\ttfamily\scriptsize]
{"permissions": {"allow": ["Bash(npm test)", "Bash(git status)"]}}
    \end{lstlisting}

    \vspace{0.5em}
    \textbf{Skip all prompts} (use with caution!):
    \begin{lstlisting}[basicstyle=\ttfamily\small]
claude --dangerously-skip-permissions
    \end{lstlisting}
    Only use this in trusted environments where you understand the risks.
\end{frame}

\begin{frame}[fragile]{Customization: CLAUDE.md}
    Create a \texttt{CLAUDE.md} file in your project root to give Claude persistent context:

    \begin{lstlisting}[basicstyle=\ttfamily\scriptsize]
# Project: My Research Analysis

## Overview
This project analyzes survey data from...

## Conventions
- Use numpy for numerical operations
- All functions should have docstrings
- Save outputs to the `results/` folder

## Important Files
- `data/raw_survey.csv` - Main dataset
- `src/analysis.py` - Core analysis functions
    \end{lstlisting}

    Claude reads this automatically when you start a session.
\end{frame}

\begin{frame}[fragile]{Model Selection}
    \textbf{Claude Code supports three model tiers:}

    \begin{lstlisting}
/model opus      # Opus 4.5 - most capable, complex reasoning
/model sonnet    # Sonnet 4 (default) - balanced, best for coding
/model haiku     # Haiku 3.5 - fastest, lightweight tasks
    \end{lstlisting}

    \vspace{1em}
    \textbf{When to use which:}
    \begin{itemize}
        \item \textbf{Opus}: Multi-file refactoring, complex debugging, large tasks
        \item \textbf{Sonnet}: Day-to-day coding, documentation, analysis
        \item \textbf{Haiku}: Quick questions, simple edits, cost-sensitive work
    \end{itemize}
\end{frame}

\begin{frame}[fragile]{Useful Commands}
    \begin{lstlisting}
/help            # See all available commands
/clear           # Clear conversation history
/compact         # Summarize conversation to clear context
/cost            # Check token usage and costs
/review          # Request a code review
    \end{lstlisting}

    \vspace{1em}
    \textbf{Configuration:}
    \begin{lstlisting}
claude config    # Manage API keys, defaults, preferences
    \end{lstlisting}
\end{frame}

\begin{frame}{Context Management}
    \textbf{What is ``context''?} The information Claude has available to understand your project: files it has read, conversation history, and command outputs.

    \vspace{1em}
    \textbf{Claude Code has a 200K token context window.} That's roughly:
    \begin{itemize}
        \item \textasciitilde150,000 words of text (a 500-page book)
        \item \textasciitilde10,000 lines of code across multiple files
        \item The entire DSGE.jl codebase we're using in Demo 2
        \item Your full PhD thesis codebase (19 files) + conversation history
    \end{itemize}

    \vspace{1em}
    \textbf{Managing context effectively:}
    \begin{itemize}
        \item Use \texttt{/compact} to summarize long conversations (frees up tokens)
        \item Reference specific files rather than ``the whole project''
        \item CLAUDE.md helps focus on what matters
    \end{itemize}
\end{frame}

\begin{frame}[fragile]{Subagents}
    Claude Code can spawn \textbf{subagents} -- separate instances that work in parallel.

    \vspace{1em}
    \textbf{How to use them:} Just ask in natural language:
    \begin{lstlisting}[basicstyle=\ttfamily\small]
"Spawn a subagent to verify that my refactored code
produces the same output as the original"
    \end{lstlisting}

    \vspace{1em}
    \textbf{Use cases:}
    \begin{itemize}
        \item Explore different parts of a codebase simultaneously
        \item Verify results independently (one agent writes, another checks)
        \item Handle complex multi-step tasks in parallel
    \end{itemize}

    \vspace{1em}
    Subagents provide ``second opinions'' and catch errors you might miss.
\end{frame}

\begin{frame}{Pro Tip: Write Prompts in a Text File}
    \textbf{Why?}
    \begin{enumerate}
        \item Prevents accidental Enter sending incomplete prompts
        \item Creates a log of your prompts for reproducibility
        \item Allows careful editing before sending
        \item Easy to reuse prompts across sessions
    \end{enumerate}

    \vspace{1em}
    \textbf{Workflow:}
    \begin{enumerate}
        \item Write prompt in \texttt{prompt.txt}
        \item Copy and paste into Claude Code interactive prompt
        \item Keep the file as documentation
    \end{enumerate}
\end{frame}

\begin{frame}{Demo 1: Linear Regression from Scratch}
    \begin{center}
        \Large
        \textbf{Goal:} Create a complete OLS implementation without sklearn
    \end{center}
\end{frame}

\begin{frame}[fragile]{Demo 1: Prompt}
    \begin{lstlisting}[basicstyle=\ttfamily\scriptsize]
Create a LinearRegression class in Python that implements
OLS regression from scratch (without sklearn). Include:

1. fit(X, y) method using closed-form OLS solution
2. predict(X) method
3. Properties: coefficients, intercept, r_squared, standard_errors
4. summary() method with formatted table (like statsmodels)
5. plot_fit(X, y) method for visualization

Use only numpy, scipy.stats, and matplotlib. Include docstrings.
    \end{lstlisting}
\end{frame}

\begin{frame}{Demo 2: Documenting Julia Code (FRBNY DSGE)}
    \begin{center}
        \Large
        \textbf{Goal:} Read and document the NY Fed's DSGE model
    \end{center}
\end{frame}

\begin{frame}[fragile]{Demo 2: Prompt}
    \begin{lstlisting}[basicstyle=\ttfamily\scriptsize]
Explore the DSGE.jl Julia codebase in
demo2_julia_documentation/DSGE.jl-main/

Give me an overview of:
- Its general purpose (what economic questions does it answer?)
- The main file structure
- Julia version and key dependencies from Project.toml
- Where to find example scripts
    \end{lstlisting}
\end{frame}

\begin{frame}{Demo 3: Refactoring PhD Codebase}
    \begin{center}
        \Large
        \textbf{Scenario:} ``2 years of thesis work. Code is a mess. Defense is next month.''

        \vspace{1em}
        \textbf{19 Python files} with hardcoded paths, duplicated functions, old\_scripts/ folder...
    \end{center}
\end{frame}

\begin{frame}[fragile]{Demo 3: Prompt}
    \begin{lstlisting}[basicstyle=\ttfamily\scriptsize]
Explore this research codebase in the messy_codebase folder.
I'm a PhD student who built this over 2 years of thesis work
and need to clean it up before my defense.

Give me an overview of:
- The overall structure and what each file does
- The data pipeline (download -> process -> merge -> analyze)
- Main issues you see with the organization
    \end{lstlisting}
\end{frame}

\begin{frame}[fragile]{Demo 3: Follow-up Prompt}
    \begin{lstlisting}[basicstyle=\ttfamily\scriptsize]
Create a refactoring plan for this codebase. Goals:
- Make it work on any computer (no hardcoded paths)
- Single source of truth for data loading
- Clear project structure with config file
- Remove dead/duplicate code
- Proper entry point that doesn't use exec()
- Keep all functionality intact
    \end{lstlisting}
\end{frame}

\begin{frame}[fragile]{Demo 3: Verification Prompt}
    \begin{lstlisting}[basicstyle=\ttfamily\scriptsize]
Spawn a subagent to review the refactored code and verify:
- All original functionality is preserved
- Paths are properly relative/configurable
- No duplicate code remains
- The pipeline can run end-to-end
    \end{lstlisting}
\end{frame}

\begin{frame}{Demo 4: Data Consolidation (Bonus)}
    \begin{center}
        \Large
        \textbf{Scenario:} Economic data from multiple sources (FRED, Michigan Survey, SPF)

        \vspace{1em}
        \normalsize
        Mixed formats (CSV, XLS, XLSX), different date conventions, varying frequencies
    \end{center}
\end{frame}

\begin{frame}[fragile]{Demo 4: Analysis Prompt}
    \begin{lstlisting}[basicstyle=\ttfamily\scriptsize]
Explore the data files in the fred/, michigan/, and spf/ folders.

For each source, analyze and report:
- What variables are available
- Date format and frequency (monthly, quarterly, etc.)
- Time range covered
- Any data quality issues (missing values, inconsistencies)

Give me a summary report before we proceed.
    \end{lstlisting}
\end{frame}

\begin{frame}[fragile]{Demo 4: Merging Options Prompt}
    \begin{lstlisting}[basicstyle=\ttfamily\scriptsize]
Based on your analysis, propose 2-3 options for merging
this data into a unified dataset. Consider:
- How to handle different frequencies
- Date alignment strategy
- Which variables to include
- How to handle the XLS/XLSX files

Recommend the best approach and explain why.
    \end{lstlisting}
\end{frame}

\begin{frame}[fragile]{Demo 4: Verification Prompt}
    \begin{lstlisting}[basicstyle=\ttfamily\scriptsize]
Spawn subagents to verify the merged dataset:
- One agent: check date alignment is correct
- One agent: verify no data was lost in the merge
- One agent: validate column names and types

Report back any issues found.
    \end{lstlisting}
\end{frame}

\begin{frame}{Best Practices}
    \begin{enumerate}
        \item \textbf{Be specific} -- ``Add error handling to load\_data()'' $>$ ``fix the code''
        \item \textbf{Iterate} -- Start simple, refine in follow-up prompts
        \item \textbf{Verify} -- Always review generated code before running
        \item \textbf{Use context} -- Reference specific files and functions
        \item \textbf{Trust but validate} -- Test outputs, check edge cases
    \end{enumerate}
\end{frame}

\begin{frame}{Limitations to Keep in Mind}
    \begin{itemize}
        \item \textbf{Not a replacement for understanding} -- You need to verify the code
        \item \textbf{Can hallucinate} -- May reference non-existent functions/libraries
        \item \textbf{Security considerations} -- Review before running system commands
        \item \textbf{Novel research} -- May not know cutting-edge methods
        \item \textbf{Liability} -- It is your research - you are responsible for it
        \item \textbf{Interprets the commands you gave only} -- Unless instructed correctly, it will ``run wild'' 
        \item \textbf{Overwhelming} -- It will produce months worth of work in minutes. 
    \end{itemize}
\end{frame}

\begin{frame}{Resources}
    \textbf{Documentation:}
    \begin{itemize}
        \item \url{https://docs.anthropic.com/en/docs/claude-code}
    \end{itemize}

    \vspace{0.5em}
    \textbf{Claude Subscription:}
    \begin{itemize}
        \item \url{https://claude.ai/} (Pro/Team plans include Claude Code access)
    \end{itemize}

    \vspace{0.5em}
    \textbf{Gemini (Free for Berkeley accounts!):}
    \begin{itemize}
        \item \url{https://gemini.google.com/}
    \end{itemize}

    \vspace{0.5em}
    \textbf{D-Lab:}
    \begin{itemize}
        \item \url{https://dlab.berkeley.edu/}
    \end{itemize}

    \vspace{1em}
    \textbf{Questions?} Let's discuss!
\end{frame}

\end{document}
